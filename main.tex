\documentclass[12pt,a4paper]{article}
\usepackage{color}
\usepackage{float}
\usepackage{graphicx}
\usepackage{indentfirst}
\usepackage{amsmath}
\usepackage{multirow}
\usepackage{url}
\usepackage{booktabs}
\def\degree{${}^{\circ}$}
\begin{document}

\vspace*{0.25cm}

\hrulefill

\thispagestyle{empty}

\begin{center}
\begin{large}
\sc{UM--SJTU Joint Institute \vspace{0.3em} \\ Introduction to Cryptography \\(VE475)}
\end{large}

\hrulefill

\vspace*{5cm}
\begin{Large}
\sc{{Project Report}}
\end{Large}

\vspace{2em}

\begin{large}
\sc{{Group 1\\
\vspace{0.5em}
bitcoin \\}}
\end{large}
\end{center}


\vfill

\begin{table}[h!]
\flushleft
\begin{tabular}{lll}
Name: Liu Niyiqiu \hspace*{2em}&
ID: 516370910118\hspace*{2em}
\\
Name: Xiang Zhiyuan \hspace*{2em}&
ID: 516370910118\hspace*{2em}
\\
Name: S \hspace*{2em}&
ID: 516370910118\hspace*{2em}
\\


\\

Date: 26 July 2019
\end{tabular}
\end{table}

\hfill

\newpage
\tableofcontents
\newpage

\section{Mining}
\subsection{Definition of Mining}
The bitcoin is a decentralized cryptocurrent. No authorities are present to authenticate each transaction. Thus the burden of verifying transactions and gathering valid transactions lies to the miners. The ultimate goal of a miner is to constitute a block by solving a mathematical problem, which will be described in the next section. To compensate the computational power spent by the miner, a reward of 12.5 bitcoins is given to the first miner that create a new block. Also, the two parties between a transaction may specify a transaction fee that will be given to the miner. 
\subsection{Mathematics of Mining}
\subsection{The Byzantine Generals' Problem}

\section{References}
\begin{enumerate}
	\item The Mathematics Behind Bitcoin, Cyril Grunspan, \url{https://webusers.imj-prg.fr/~ricardo.perez-marco/blockchain/BitcoinP7.pdf}
\end{enumerate}

\end{document}
